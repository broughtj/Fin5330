\documentclass[11pt]{article}

\usepackage{graphicx}
\usepackage{lscape}
\usepackage{amsmath}
\usepackage{mathrsfs}
\usepackage[pdftex]{color}
\usepackage{textcomp}
\usepackage{hyperref}

\setlength{\parindent}{0in}

\hypersetup{
    bookmarks=true,                       % show bookmarks bar?
    unicode=false,                        % non-Latin characters in Acrobat’s bookmarks
    pdftoolbar=true,                      % show Acrobat’s toolbar?
    pdfmenubar=true,                      % show Acrobat’s menu?
    pdffitwindow=false,                   % window fit to page when opened
    pdfstartview={FitH},                  % fits the width of the page to the window
    pdftitle={My title},                  % title
    pdfauthor={Author},                   % author
    pdfsubject={Subject},                 % subject of the document
    pdfcreator={Creator},                 % creator of the document
    pdfproducer={Producer},               % producer of the document
    pdfkeywords={keyword1} {key2} {key3}, % list of keywords
    pdfnewwindow=true,                    % links in new window
    colorlinks=true,                      % false: boxed links; true: colored links
    linkcolor=red,                        % color of internal links
    citecolor=green,                      % color of links to bibliography
    filecolor=magenta,                    % color of file links
    urlcolor=blue                         % color of external links
}

\begin{document}\pagestyle{empty}

\textbf{Finance 5330, Spring 2020} \\
\textbf{Project I -  Description}     \\

\bigskip
This project is based on the research article \textit{Illuminating the Profitability of Pairs Trading:
A Test of the Relative Pricing Efficiency of Markets for Water Utility Stocks} by Gutierrez and Tse (GT).
GT examine the profitability of the basic pairs trading strategy. This strategy relies on identifying
pairs of assets that are cointegrated. GT study stocks in the water utility industry in the hopes that 
pairs of them will be cointegrated. 

\bigskip
Your assignment is to reproduce Exhibit 2 Panel A and Panel B, as well as Exhibit 3 (all panels) for
three stocks that they choose. The deliverable is a Jupyter notebook with Julia or Python (or R, or 
whatever) code that carries out the calculations. You should make tables of the results in your document
and write several paragraphs of prose to explain the results. 

\bigskip
This project does not require you to carry out the analysis for the actual trading strategy. Instead,
you are required to discuss how the results you found in your analysis of cointegration and 
error-correction enable you to form such strategies. Outline a strategy of how you would use the
econometric foundation from your analysis to set up and carry out pairs trading strategies in real
life. What factors need to be considered when going from econometric models to real life trading?
What data, systems, or other factors need to be considered? How does having an econometric foundation
aid the process? What other statistical or econometric methods or procedures would aid your plan?
What concerns about the process do you forsee as the portfolio manager?

\end{document}


