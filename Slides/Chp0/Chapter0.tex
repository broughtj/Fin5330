\documentclass[10pt,ignorenonframetext,,aspectratio=149]{beamer}
\setbeamertemplate{caption}[numbered]
\setbeamertemplate{caption label separator}{: }
\setbeamercolor{caption name}{fg=normal text.fg}
\usepackage{lmodern}
\usepackage{amssymb,amsmath}
\usepackage{ifxetex,ifluatex}
\usepackage{fixltx2e} % provides \textsubscript
\ifnum 0\ifxetex 1\fi\ifluatex 1\fi=0 % if pdftex
  \usepackage[T1]{fontenc}
  \usepackage[utf8]{inputenc}
\else % if luatex or xelatex
  \ifxetex
    \usepackage{mathspec}
  \else
    \usepackage{fontspec}
  \fi
  \defaultfontfeatures{Ligatures=TeX,Scale=MatchLowercase}
  \newcommand{\euro}{€}
\fi
% use upquote if available, for straight quotes in verbatim environments
\IfFileExists{upquote.sty}{\usepackage{upquote}}{}
% use microtype if available
\IfFileExists{microtype.sty}{%
\usepackage{microtype}
\UseMicrotypeSet[protrusion]{basicmath} % disable protrusion for tt fonts
}{}
\usepackage{color}
\usepackage{fancyvrb}
\newcommand{\VerbBar}{|}
\newcommand{\VERB}{\Verb[commandchars=\\\{\}]}
\DefineVerbatimEnvironment{Highlighting}{Verbatim}{commandchars=\\\{\}}
% Add ',fontsize=\small' for more characters per line
\usepackage{framed}
\definecolor{shadecolor}{RGB}{248,248,248}
\newenvironment{Shaded}{\begin{snugshade}}{\end{snugshade}}
\newcommand{\KeywordTok}[1]{\textcolor[rgb]{0.13,0.29,0.53}{\textbf{#1}}}
\newcommand{\DataTypeTok}[1]{\textcolor[rgb]{0.13,0.29,0.53}{#1}}
\newcommand{\DecValTok}[1]{\textcolor[rgb]{0.00,0.00,0.81}{#1}}
\newcommand{\BaseNTok}[1]{\textcolor[rgb]{0.00,0.00,0.81}{#1}}
\newcommand{\FloatTok}[1]{\textcolor[rgb]{0.00,0.00,0.81}{#1}}
\newcommand{\ConstantTok}[1]{\textcolor[rgb]{0.00,0.00,0.00}{#1}}
\newcommand{\CharTok}[1]{\textcolor[rgb]{0.31,0.60,0.02}{#1}}
\newcommand{\SpecialCharTok}[1]{\textcolor[rgb]{0.00,0.00,0.00}{#1}}
\newcommand{\StringTok}[1]{\textcolor[rgb]{0.31,0.60,0.02}{#1}}
\newcommand{\VerbatimStringTok}[1]{\textcolor[rgb]{0.31,0.60,0.02}{#1}}
\newcommand{\SpecialStringTok}[1]{\textcolor[rgb]{0.31,0.60,0.02}{#1}}
\newcommand{\ImportTok}[1]{#1}
\newcommand{\CommentTok}[1]{\textcolor[rgb]{0.56,0.35,0.01}{\textit{#1}}}
\newcommand{\DocumentationTok}[1]{\textcolor[rgb]{0.56,0.35,0.01}{\textbf{\textit{#1}}}}
\newcommand{\AnnotationTok}[1]{\textcolor[rgb]{0.56,0.35,0.01}{\textbf{\textit{#1}}}}
\newcommand{\CommentVarTok}[1]{\textcolor[rgb]{0.56,0.35,0.01}{\textbf{\textit{#1}}}}
\newcommand{\OtherTok}[1]{\textcolor[rgb]{0.56,0.35,0.01}{#1}}
\newcommand{\FunctionTok}[1]{\textcolor[rgb]{0.00,0.00,0.00}{#1}}
\newcommand{\VariableTok}[1]{\textcolor[rgb]{0.00,0.00,0.00}{#1}}
\newcommand{\ControlFlowTok}[1]{\textcolor[rgb]{0.13,0.29,0.53}{\textbf{#1}}}
\newcommand{\OperatorTok}[1]{\textcolor[rgb]{0.81,0.36,0.00}{\textbf{#1}}}
\newcommand{\BuiltInTok}[1]{#1}
\newcommand{\ExtensionTok}[1]{#1}
\newcommand{\PreprocessorTok}[1]{\textcolor[rgb]{0.56,0.35,0.01}{\textit{#1}}}
\newcommand{\AttributeTok}[1]{\textcolor[rgb]{0.77,0.63,0.00}{#1}}
\newcommand{\RegionMarkerTok}[1]{#1}
\newcommand{\InformationTok}[1]{\textcolor[rgb]{0.56,0.35,0.01}{\textbf{\textit{#1}}}}
\newcommand{\WarningTok}[1]{\textcolor[rgb]{0.56,0.35,0.01}{\textbf{\textit{#1}}}}
\newcommand{\AlertTok}[1]{\textcolor[rgb]{0.94,0.16,0.16}{#1}}
\newcommand{\ErrorTok}[1]{\textcolor[rgb]{0.64,0.00,0.00}{\textbf{#1}}}
\newcommand{\NormalTok}[1]{#1}

% Comment these out if you don't want a slide with just the
% part/section/subsection/subsubsection title:
\AtBeginPart{
  \let\insertpartnumber\relax
  \let\partname\relax
  \frame{\partpage}
}
\AtBeginSection{
  \let\insertsectionnumber\relax
  \let\sectionname\relax
  \frame{\sectionpage}
}
\AtBeginSubsection{
  \let\insertsubsectionnumber\relax
  \let\subsectionname\relax
  \frame{\subsectionpage}
}

\setlength{\emergencystretch}{3em}  % prevent overfull lines
\providecommand{\tightlist}{%
  \setlength{\itemsep}{0pt}\setlength{\parskip}{0pt}}
\setcounter{secnumdepth}{0}

\title{Finance 5330 - Financial Econometrics}
\subtitle{Prices, Returns and Price Discovery}
\author{Tyler J. Brough}
\date{}

%% Here's everything I added.
%%--------------------------

\usepackage{graphicx}
\usepackage{rotating}
%\setbeamertemplate{caption}[numbered]
\usepackage{hyperref}
\usepackage{caption}
\usepackage[normalem]{ulem}
%\mode<presentation>
\usepackage{wasysym}
%\usepackage{amsmath}


% Get rid of navigation symbols.
%-------------------------------
\setbeamertemplate{navigation symbols}{}

% Optional institute tags and titlegraphic.
% Do feel free to change the titlegraphic if you don't want it as a Markdown field.
%----------------------------------------------------------------------------------
\institute{Department of Finance and Economics}

% \titlegraphic{\includegraphics[width=0.3\paperwidth]{\string~/Dropbox/teaching/clemson-academic.png}} % <-- if you want to know what this looks like without it as a Markdown field. 
% -----------------------------------------------------------------------------------------------------
\titlegraphic{\includegraphics[width=0.3\paperwidth]{\string~./images/vertical-logo-blue.png}}

% Some additional title page adjustments.
%----------------------------------------
\setbeamertemplate{title page}[empty]
%\date{}
\setbeamerfont{subtitle}{size=\small}

\setbeamercovered{transparent}

% Some optional colors. Change or add as you see fit.
%---------------------------------------------------
\definecolor{clemsonpurple}{HTML}{522D80}
% \definecolor{clemsonorange}{HTML}{EA6A20}
 \definecolor{clemsonorange}{HTML}{F66733}
\definecolor{uiucblue}{HTML}{003C7D}
\definecolor{uiucorange}{HTML}{F47F24}
%\definecolor{usublue}{HTML}{2D3280}
%\definecolor{usublue}{HTML}{000066}
%\definecolor{usublue}{HTML}{273849}
%\definecolor{usublue}{HTML}{78281F}
%\definecolor{usublue}{HTML}{A04000}


%\definecolor{usublue}{HTML}{3498DB}
%\definecolor{bluegray}{HTML}{334760}
\definecolor{usublue}{HTML}{C0392B}
\definecolor{bluegray}{HTML}{34495E}

\definecolor{gray}{HTML}{446280}

% Some optional color adjustments to Beamer. Change as you see fit.
%------------------------------------------------------------------
%\setbeamercolor{frametitle}{fg=clemsonpurple,bg=white}
\setbeamercolor{frametitle}{fg=usublue,bg=white}
\setbeamercolor{title}{fg=usublue,bg=white}
\setbeamercolor{local structure}{fg=usublue}
\setbeamercolor{section in toc}{fg=usublue,bg=white}
% \setbeamercolor{subsection in toc}{fg=clemsonorange,bg=white}
\setbeamercolor{footline}{fg=usublue!50, bg=white}
\setbeamercolor{block title}{fg=gray,bg=white}


\let\Tiny=\tiny


% Sections and subsections should not get their own damn slide.
%--------------------------------------------------------------
\AtBeginPart{}
\AtBeginSection{}
\AtBeginSubsection{}
\AtBeginSubsubsection{}

% Suppress some of Markdown's weird default vertical spacing.
%------------------------------------------------------------
\setlength{\emergencystretch}{0em}  % prevent overfull lines
\setlength{\parskip}{0pt}


% Allow for those simple two-tone footlines I like. 
% Edit the colors as you see fit.
%--------------------------------------------------
\defbeamertemplate*{footline}{my footline}{%
    \ifnum\insertpagenumber=1
    \hbox{%
        \begin{beamercolorbox}[wd=\paperwidth,ht=.8ex,dp=1ex,center]{}%
      % empty environment to raise height
        \end{beamercolorbox}%
    }%
    \vskip0pt%
    \else%
        \Tiny{%
            \hfill%
		\vspace*{1pt}%
            \insertframenumber/\inserttotalframenumber \hspace*{0.1cm}%
            \newline%
            \color{usublue}{\rule{\paperwidth}{0.4mm}}\newline%
            \color{bluegray}{\rule{\paperwidth}{.4mm}}%
        }%
    \fi%
}

% Various cosmetic things, though I must confess I forget what exactly these do and why I included them.
%-------------------------------------------------------------------------------------------------------
\setbeamercolor{structure}{fg=blue}
\setbeamercolor{local structure}{parent=structure}
\setbeamercolor{item projected}{parent=item,use=item,fg=usublue,bg=white}
\setbeamercolor{enumerate item}{parent=item}

% Adjust some item elements. More cosmetic things.
%-------------------------------------------------
\setbeamertemplate{itemize item}{\color{usublue}$\bullet$}
\setbeamertemplate{itemize subitem}{\color{usublue}\scriptsize{$\bullet$}}
\setbeamertemplate{itemize/enumerate body end}{\vspace{.6\baselineskip}} % So I'm less inclined to use \medskip and \bigskip in Markdown.

% Automatically center images
% ---------------------------
% Note: this is for ![](image.png) images
% Use "fig.align = "center" for R chunks

\usepackage{etoolbox}

\AtBeginDocument{%
  \letcs\oig{@orig\string\includegraphics}%
  \renewcommand<>\includegraphics[2][]{%
    \only#3{%
      {\centering\oig[{#1}]{#2}\par}%
    }%
  }%
}

% I think I've moved to xelatex now. Here's some stuff for that.
% --------------------------------------------------------------
% I could customize/generalize this more but the truth is it works for my circumstances.

\ifxetex
\setbeamerfont{title}{family=\fontspec{serif}}
\setbeamerfont{frametitle}{family=\fontspec{serif}}
\usepackage[font=small,skip=0pt]{caption}
 \else
 \fi

% Okay, and begin the actual document...

\begin{document}
\frame{\titlepage}

\begin{frame}

\end{frame}

\section{Introduction to Financial Data: Price and
Returns}\label{introduction-to-financial-data-price-and-returns}

\begin{frame}{Asset Returns}

\textbf{One-Period Simple Return}

\begin{itemize}
\item
  Hold asset from period \(t-1\) to \(t\)
\item
  Simple Gross Return:
\end{itemize}

\[
1 + R_{t} = \frac{P_{t}}{P_{t-1}}
\]

or

\[
P_{t} = P_{t-1} (1 + R_{t})
\]

\end{frame}

\begin{frame}{Multiperiod Simple Return}

\begin{itemize}
\tightlist
\item
  Holding the asset \(k\) period between dates \(t-k\) and \(t\) gives a
  \(k\mbox{-period}\) gross return:
\end{itemize}

\[
\begin{aligned}
1 + R_{t}[k] &= \frac{P_{t}}{P_{t-k}} = \frac{P_{t}}{P_{t-1}} \frac{P_{t-1}}{P_{t-2}} \cdots \frac{P_{t-k+1}}{P_{t-k}} \\
             &= (1 + R_{t}) (1 + R_{t-1}) \cdots (1 + R_{t-k+1}) \\
             &= \prod_{j=0}^{k-1} (1 + R_{t-j})
\end{aligned}
\]

\begin{itemize}
\item
  \textbf{\emph{NB:}} the \(k\mbox{-period}\) simple gross return is the
  product of the \(k\) one-period simple gross returns. This is called a
  compound return.
\item
  The \(k\mbox{-period}\) simple net return is:
  \(R_{t}[k] = \frac{P_{t} - P_{t-k}}{P_{t-k}}\)
\end{itemize}

\end{frame}

\begin{frame}{Annualized Returns}

\begin{itemize}
\item
  To facilitate comparison we usually standardize to a given duration
  (annual, monthly, quarterly, etc)
\item
  The annual return is
\end{itemize}

\[
\mbox{Annualized}\{R_{t}[k]\} = \left[ \prod_{j=0}^{k-1} (1 + R_{t-j}) \right]^{1/k} - 1
\]

\begin{itemize}
\item
  \textbf{\emph{NB:}} this is a geometric mean of the \(k\) one-period
  simple gross returns
\item
  It can also be computed as
\end{itemize}

\[
\mbox{Annualized}\{R_{t}[k]\} = \exp{\left[\frac{1}{k} \sum\limits_{j=0}^{k-1} \ln{(1 + R_{t-j})} \right]} - 1
\]

\end{frame}

\begin{frame}{Continuous Compounding}

For continuous compounding we can use the following formula:

\[
A = C \exp{(r \times n)}
\]

where

\begin{itemize}
\item
  \(A =\) net asset value
\item
  \(C =\) initial capital
\item
  \(r =\) interest rate per annum
\item
  \(n =\) number of years
\end{itemize}

With this in mind, we can also define the \textbf{\emph{present value}}
relation:

\[
C = A \exp{(-r \times n)}
\]

\end{frame}

\begin{frame}{Continuously Compounded Return}

\begin{itemize}
\tightlist
\item
  The natural log of the simple gross return
\end{itemize}

\[
r_{t} = \ln{(1 + R_{t})} = \ln{\frac{P_{t}}{P_{t-1}}} = \ln{P_{t}} - \ln{P_{t-1}}
\]

\begin{itemize}
\tightlist
\item
  Advantages

  \begin{itemize}
  \tightlist
  \item
    Multiperiod returns
  \end{itemize}
\end{itemize}

\[
\begin{aligned}
r_{t}[k] &= \ln{(1 + R_{t}[k])} \\
         &= \ln{\left[(1 + R_{t}) (1 + R_{t-1}) \cdots (1 + R_{t-k+1})\right]} \\
         &= \ln{(1 + R_{t})} + \ln{(1 + R_{t-1})} + \cdots + \ln{(1 + R_{t-k+1})} \\
         &= r_{t} + r_{t-1} + \cdots + r_{t-k+1} 
\end{aligned}
\]

\end{frame}

\begin{frame}[fragile]{Continously Compounded Return (Continued)}

\begin{itemize}
\item
  Symmetric up/down moves
\item
  Set up some imaginary prices
\end{itemize}

\begin{Shaded}
\begin{Highlighting}[]
\NormalTok{p1 }\OperatorTok{=} \FloatTok{100.0}
\NormalTok{p2 }\OperatorTok{=} \FloatTok{105.0}
\NormalTok{p3 }\OperatorTok{=} \FloatTok{100.0}
\end{Highlighting}
\end{Shaded}

\end{frame}

\begin{frame}[fragile]{Continously Compounded Return (Continued)}

\begin{itemize}
\tightlist
\item
  The simple net return
\end{itemize}

\begin{Shaded}
\begin{Highlighting}[]
\NormalTok{R2 }\OperatorTok{=}\NormalTok{ p2}\OperatorTok{/}\NormalTok{p1 }\OperatorTok{-} \FloatTok{1.0}
\BuiltInTok{print}\NormalTok{(R2)}
\end{Highlighting}
\end{Shaded}

\begin{verbatim}
## 0.050000000000000044
\end{verbatim}

\begin{Shaded}
\begin{Highlighting}[]
\NormalTok{R3 }\OperatorTok{=}\NormalTok{ p3}\OperatorTok{/}\NormalTok{p2 }\OperatorTok{-} \FloatTok{1.0}
\BuiltInTok{print}\NormalTok{(R3)}
\end{Highlighting}
\end{Shaded}

\begin{verbatim}
## -0.04761904761904767
\end{verbatim}

\end{frame}

\begin{frame}[fragile]{Continously Compounded Return (Continued)}

\begin{itemize}
\tightlist
\item
  Continously compounded prices
\end{itemize}

\begin{Shaded}
\begin{Highlighting}[]
\ImportTok{import}\NormalTok{ numpy }\ImportTok{as}\NormalTok{ np}
\NormalTok{r2 }\OperatorTok{=}\NormalTok{ np.log(p2) }\OperatorTok{-}\NormalTok{ np.log(p1)}
\BuiltInTok{print}\NormalTok{(r2)}
\end{Highlighting}
\end{Shaded}

\begin{verbatim}
## 0.04879016416943127
\end{verbatim}

\begin{Shaded}
\begin{Highlighting}[]
\NormalTok{r3 }\OperatorTok{=}\NormalTok{ np.log(p3) }\OperatorTok{-}\NormalTok{ np.log(p2)}
\BuiltInTok{print}\NormalTok{(r3)}
\end{Highlighting}
\end{Shaded}

\begin{verbatim}
## -0.04879016416943127
\end{verbatim}

\end{frame}

\begin{frame}{Portfolio Returns}

\[
R_{p,t} = \sum\limits_{i=1}^{N} \omega_{i} R_{i,t}
\]

where

\begin{itemize}
\item
  \(p =\) index representing the portfolio
\item
  \(N =\) number of assets in the portfolio
\item
  \(R_{i,t} =\) period \(t\) simple net return on aset \(i\)
\item
  \(\omega_{i} =\) weight on asset \(i\)
\end{itemize}

\textbf{\emph{NB:}} continously compounded portfolio returns do not have
this convenient property, but for simple returns \(R_{i,t}\) small in
magnitude, we have

\[
r_{p,t} \approx \sum\limits_{i=1}^{N} \omega_{i} r_{i,t}
\]

\end{frame}

\begin{frame}{Accounting for Dividend Payments}

\begin{itemize}
\tightlist
\item
  We often have to account for dividends
\end{itemize}

\[
R_{t} = \frac{(P_{t} + D_{t})}{P_{t-1}} - 1
\]

\[
r_{t} = \ln{(P_{t} + D_{t})} - \ln{P_{t-1}}
\]

\end{frame}

\begin{frame}{Excess Return}

\[
Z_{t} = R_{t} - R_{0,t}
\]

where \(R_{0,t}\) is a reference asset (e.g.~the risk-free rate)

\[
z_{t} = r_{t} - r_{0,t}
\]

\textbf{\emph{NB:}} Excess return can be thought of as the payoff to an
arbitrage strategy that goes long the asset and short the reference rate
with no initial investment.

\end{frame}

\begin{frame}{Summary of Relationships}

\begin{itemize}
\item
  \(r_{t} = \ln{(1 + R_{t})} \implies R_{t} = \exp{(r_{t})} - 1\)
\item
  If \(R_{t}\) and \(r_{t}\) are in percentages then

  \begin{itemize}
  \tightlist
  \item
    \(r_{t} = 100 \ln{\left(1 + \frac{R_{t}}{100}\right)}\)
  \item
    \(R_{t} = 100 \left(e^{r_{t}/100} - 1 \right)\)
  \end{itemize}
\item
  Temporal aggregation produces

  \begin{itemize}
  \tightlist
  \item
    \(1 + R_{t}[k] = (1 + R_{t}) (1 + R_{t-1}) \cdots (1 + R_{t-k+1})\)
  \item
    \(r_{t}[k] = r_{t} + r_{t-1} + \cdots + r_{t-k+1}\)
  \end{itemize}
\item
  Present and future values:

  \begin{itemize}
  \tightlist
  \item
    \(A = C \exp{(r \times n)}\)
  \item
    \(C = A \exp{(-r \times n)}\)
  \end{itemize}
\end{itemize}

\end{frame}


\section[]{}
\frame{\small \frametitle{Table of Contents}
\tableofcontents}
\end{document}
