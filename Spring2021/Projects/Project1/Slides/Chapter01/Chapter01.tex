\documentclass[10pt,ignorenonframetext,,aspectratio=149]{beamer}
\setbeamertemplate{caption}[numbered]
\setbeamertemplate{caption label separator}{: }
\setbeamercolor{caption name}{fg=normal text.fg}
\usepackage{lmodern}
\usepackage{amssymb,amsmath}
\usepackage{ifxetex,ifluatex}
\usepackage{fixltx2e} % provides \textsubscript
\ifnum 0\ifxetex 1\fi\ifluatex 1\fi=0 % if pdftex
  \usepackage[T1]{fontenc}
  \usepackage[utf8]{inputenc}
\else % if luatex or xelatex
  \ifxetex
    \usepackage{mathspec}
  \else
    \usepackage{fontspec}
  \fi
  \defaultfontfeatures{Ligatures=TeX,Scale=MatchLowercase}
  \newcommand{\euro}{€}
\fi
% use upquote if available, for straight quotes in verbatim environments
\IfFileExists{upquote.sty}{\usepackage{upquote}}{}
% use microtype if available
\IfFileExists{microtype.sty}{%
\usepackage{microtype}
\UseMicrotypeSet[protrusion]{basicmath} % disable protrusion for tt fonts
}{}

% Comment these out if you don't want a slide with just the
% part/section/subsection/subsubsection title:
\AtBeginPart{
  \let\insertpartnumber\relax
  \let\partname\relax
  \frame{\partpage}
}
\AtBeginSection{
  \let\insertsectionnumber\relax
  \let\sectionname\relax
  \frame{\sectionpage}
}
\AtBeginSubsection{
  \let\insertsubsectionnumber\relax
  \let\subsectionname\relax
  \frame{\subsectionpage}
}

\setlength{\emergencystretch}{3em}  % prevent overfull lines
\providecommand{\tightlist}{%
  \setlength{\itemsep}{0pt}\setlength{\parskip}{0pt}}
\setcounter{secnumdepth}{0}

\title{Chapter 1 - Financial Asset Prices and Returns}
\subtitle{Finance 5330: Financial Econometrics}
\author{Tyler J. Brough}
\date{}

%% Here's everything I added.
%%--------------------------

\usepackage{graphicx}
\usepackage{rotating}
%\setbeamertemplate{caption}[numbered]
\usepackage{hyperref}
\usepackage{caption}
\usepackage[normalem]{ulem}
%\mode<presentation>
\usepackage{wasysym}
%\usepackage{amsmath}


% Get rid of navigation symbols.
%-------------------------------
\setbeamertemplate{navigation symbols}{}

% Optional institute tags and titlegraphic.
% Do feel free to change the titlegraphic if you don't want it as a Markdown field.
%----------------------------------------------------------------------------------
\institute{Department of Finance and Economics}

% \titlegraphic{\includegraphics[width=0.3\paperwidth]{\string~/Dropbox/teaching/clemson-academic.png}} % <-- if you want to know what this looks like without it as a Markdown field. 
% -----------------------------------------------------------------------------------------------------
\titlegraphic{\includegraphics[width=0.3\paperwidth]{\string~./images/vertical-logo-blue.png}}

% Some additional title page adjustments.
%----------------------------------------
\setbeamertemplate{title page}[empty]
%\date{}
\setbeamerfont{subtitle}{size=\small}

\setbeamercovered{transparent}

% Some optional colors. Change or add as you see fit.
%---------------------------------------------------
\definecolor{clemsonpurple}{HTML}{522D80}
% \definecolor{clemsonorange}{HTML}{EA6A20}
 \definecolor{clemsonorange}{HTML}{F66733}
\definecolor{uiucblue}{HTML}{003C7D}
\definecolor{uiucorange}{HTML}{F47F24}
%\definecolor{usublue}{HTML}{2D3280}
%\definecolor{usublue}{HTML}{000066}
%\definecolor{usublue}{HTML}{273849}
%\definecolor{usublue}{HTML}{78281F}
%\definecolor{usublue}{HTML}{A04000}


%\definecolor{usublue}{HTML}{3498DB}
%\definecolor{bluegray}{HTML}{334760}
\definecolor{usublue}{HTML}{C0392B}
\definecolor{bluegray}{HTML}{34495E}

\definecolor{gray}{HTML}{446280}

% Some optional color adjustments to Beamer. Change as you see fit.
%------------------------------------------------------------------
%\setbeamercolor{frametitle}{fg=clemsonpurple,bg=white}
\setbeamercolor{frametitle}{fg=usublue,bg=white}
\setbeamercolor{title}{fg=usublue,bg=white}
\setbeamercolor{local structure}{fg=usublue}
\setbeamercolor{section in toc}{fg=usublue,bg=white}
% \setbeamercolor{subsection in toc}{fg=clemsonorange,bg=white}
\setbeamercolor{footline}{fg=usublue!50, bg=white}
\setbeamercolor{block title}{fg=gray,bg=white}


\let\Tiny=\tiny


% Sections and subsections should not get their own damn slide.
%--------------------------------------------------------------
\AtBeginPart{}
\AtBeginSection{}
\AtBeginSubsection{}
\AtBeginSubsubsection{}

% Suppress some of Markdown's weird default vertical spacing.
%------------------------------------------------------------
\setlength{\emergencystretch}{0em}  % prevent overfull lines
\setlength{\parskip}{0pt}


% Allow for those simple two-tone footlines I like. 
% Edit the colors as you see fit.
%--------------------------------------------------
\defbeamertemplate*{footline}{my footline}{%
    \ifnum\insertpagenumber=1
    \hbox{%
        \begin{beamercolorbox}[wd=\paperwidth,ht=.8ex,dp=1ex,center]{}%
      % empty environment to raise height
        \end{beamercolorbox}%
    }%
    \vskip0pt%
    \else%
        \Tiny{%
            \hfill%
		\vspace*{1pt}%
            \insertframenumber/\inserttotalframenumber \hspace*{0.1cm}%
            \newline%
            \color{usublue}{\rule{\paperwidth}{0.4mm}}\newline%
            \color{bluegray}{\rule{\paperwidth}{.4mm}}%
        }%
    \fi%
}

% Various cosmetic things, though I must confess I forget what exactly these do and why I included them.
%-------------------------------------------------------------------------------------------------------
\setbeamercolor{structure}{fg=blue}
\setbeamercolor{local structure}{parent=structure}
\setbeamercolor{item projected}{parent=item,use=item,fg=usublue,bg=white}
\setbeamercolor{enumerate item}{parent=item}

% Adjust some item elements. More cosmetic things.
%-------------------------------------------------
\setbeamertemplate{itemize item}{\color{usublue}$\bullet$}
\setbeamertemplate{itemize subitem}{\color{usublue}\scriptsize{$\bullet$}}
\setbeamertemplate{itemize/enumerate body end}{\vspace{.6\baselineskip}} % So I'm less inclined to use \medskip and \bigskip in Markdown.

% Automatically center images
% ---------------------------
% Note: this is for ![](image.png) images
% Use "fig.align = "center" for R chunks

\usepackage{etoolbox}

\AtBeginDocument{%
  \letcs\oig{@orig\string\includegraphics}%
  \renewcommand<>\includegraphics[2][]{%
    \only#3{%
      {\centering\oig[{#1}]{#2}\par}%
    }%
  }%
}

% I think I've moved to xelatex now. Here's some stuff for that.
% --------------------------------------------------------------
% I could customize/generalize this more but the truth is it works for my circumstances.

\ifxetex
\setbeamerfont{title}{family=\fontspec{serif}}
\setbeamerfont{frametitle}{family=\fontspec{serif}}
\usepackage[font=small,skip=0pt]{caption}
 \else
 \fi

% Okay, and begin the actual document...

\begin{document}
\frame{\titlepage}

\hypertarget{notes-on-chapter-01---financial-asset-prices-and-returns}{%
\section{Notes on Chapter 01 - Financial Asset Prices and
Returns}\label{notes-on-chapter-01---financial-asset-prices-and-returns}}

\begin{frame}{Section 1.1: What is Financial Econometrics?}
\protect\hypertarget{section-1.1-what-is-financial-econometrics}{}
\begin{itemize}
\item
  No simple definition
\item
  Data analysis for finance/economics (we will be reading Tukey's paper
  later)
\item
  Empirical implementation of financial models (ex: CAPM)
\item
  Methods of estimation and inference
\item
  Forecasting, policy analysis, academic understanding of financial
  market phenomena
\item
  Draws on finance/economics, probability, statistics, applied math
\item
  Connections to ML \& AI
\item
  Basics: Finance Theory + Data Analysis
\end{itemize}
\end{frame}

\begin{frame}{Section 1.2: Financial Assets}
\protect\hypertarget{section-1.2-financial-assets}{}
\begin{itemize}
\item
  Fixed income
\item
  Equity
\item
  Derivatives
\item
  Cash flows generated from these securities/contracts (streams of cash
  flows)
\end{itemize}

\vspace{10mm}

\begin{itemize}
\tightlist
\item
  Cash: claim on stream of services that it can secure by virtue of its
  role as a medium of exchange

  \begin{itemize}
  \tightlist
  \item
    Ludwig von Mises on the evolution of money
  \item
    Cash is a kind of derivative security that derives its value from
    the opportunity cost of goods and services
  \item
    Exchange rates between currencies (foreign exchange markets are the
    largest financial markets in the world)
  \end{itemize}
\end{itemize}
\end{frame}

\begin{frame}{Fixed-Income Securities}
\protect\hypertarget{fixed-income-securities}{}
\begin{itemize}
\tightlist
\item
  Two streams of cash flows

  \begin{itemize}
  \tightlist
  \item
    Stream of coupon payments made at regular fixed intervals
  \item
    The eventual return of principal at maturity
  \item
    Financial innovation is a major factor in these markets
  \item
    (the original term fixed-income came from the simplest forms.
    Subsequent forms are much more sophisticated)
  \end{itemize}
\item
  Money Markets

  \begin{itemize}
  \tightlist
  \item
    Short-term, very liquid

    \begin{itemize}
    \tightlist
    \item
      Treasury bills: simplest form of government debt (3,6,9 month
      maturities \textbar{} pure discount bonds)
    \end{itemize}
  \item
    Eurodollar deposits: deposits of US banks held in financial
    institutions outside the US denominated in USD.
  \end{itemize}
\end{itemize}
\end{frame}

\begin{frame}{}
\protect\hypertarget{section}{}
\begin{itemize}
\tightlist
\item
  Bond markets

  \begin{itemize}
  \tightlist
  \item
    Government bonds (e.g.~US Treasury) (often zero-coupon or pure
    discount)
  \item
    Corporate bonds (e.g.~CAT)

    \begin{itemize}
    \tightlist
    \item
      Typically coupon-paying bonds
    \end{itemize}
  \end{itemize}
\item
  Equity Securities

  \begin{itemize}
  \tightlist
  \item
    Common stock: give the owner an equity stake in the assets of the
    company and it's earnings

    \begin{itemize}
    \tightlist
    \item
      (call option on assets w/ K = face value of liabilities)
    \end{itemize}
  \item
    Dividends: payments representing distribution of company earnings
  \item
    Dividend yield: \$ amount per share, or as a percentage of current
    market price
  \end{itemize}
\end{itemize}
\end{frame}


%\section[]{}
%\frame{\small \frametitle{Table of Contents}
%\tableofcontents}
\end{document}
